\documentclass[a4paper]{book}
\usepackage[times,inconsolata,hyper]{Rd}
\usepackage{makeidx}
\usepackage[utf8,latin1]{inputenc}
% \usepackage{graphicx} % @USE GRAPHICX@
\makeindex{}
\begin{document}
\chapter*{}
\begin{center}
{\textbf{\huge Package `makehams'}}
\par\bigskip{\large \today}
\end{center}
\begin{description}
\raggedright{}
\item[Title]\AsIs{Ultimate Select Survival Model (AMLCR)}
\item[Version]\AsIs{1.0}
\item[Description]\AsIs{Implements Makeham's Law (with variable select period), De Moivre's Law, CFM as
well as provides various actuarial functions such as moments of insurances and annuities.}
\item[Depends]\AsIs{R (>= 2.1.1)}
\item[License]\AsIs{GPL-2}
\item[LazyData]\AsIs{true}
\end{description}
\Rdcontents{\R{} topics documented:}
\inputencoding{utf8}
\HeaderA{annx}{EPV of Annuity}{annx}
%
\begin{Description}\relax
Calculates the Expected Presented Value of various annuities
\end{Description}
%
\begin{Usage}
\begin{verbatim}
annx(x = gl.g(x), s = 0, i = gl.g(i), m = 1, n = gl.g(w) - x, c = 0,
  e = 1, mt = 1)
\end{verbatim}
\end{Usage}
%
\begin{Arguments}
\begin{ldescription}
\item[\code{x}] the current age

\item[\code{s}] the select used so far

\item[\code{i}] the interest rate

\item[\code{m}] the compounding frequency

\item[\code{n}] the length of the term

\item[\code{c}] indicator of continuous (1 if continuous)

\item[\code{e}] indicator of endowment (NOTE of an annuity should always be 1)

\item[\code{mt}] the moment of the insurance
\end{ldescription}
\end{Arguments}
%
\begin{Details}\relax
By default calculates the first moment of discrete, whole life annuity due
\end{Details}
\inputencoding{utf8}
\HeaderA{Ax}{EPV of Insurance}{Ax}
%
\begin{Description}\relax
Calculates the Expected Presented Value of various insurances
\end{Description}
%
\begin{Usage}
\begin{verbatim}
Ax(x = gl.g(x), s = 0, i = gl.g(i), m = 1, n = gl.g(w) - x, c = 0,
  e = 0, mt = 1)
\end{verbatim}
\end{Usage}
%
\begin{Arguments}
\begin{ldescription}
\item[\code{x}] the current age

\item[\code{s}] the select used so far

\item[\code{i}] the interest rate

\item[\code{m}] the compounding frequency

\item[\code{n}] the length of the term

\item[\code{c}] indicator of continuous (1 if continuous)

\item[\code{e}] indicator of endowment (1 if endowment)

\item[\code{mt}] the moment of the insurance
\end{ldescription}
\end{Arguments}
%
\begin{Details}\relax
By default calculates first moment of discrete, whole life insurance
\end{Details}
\inputencoding{utf8}
\HeaderA{cfm}{Change Survival Model to CFM}{cfm}
%
\begin{Description}\relax
Changes parameters and force of mortality function to use constant force of mortality
\end{Description}
%
\begin{Usage}
\begin{verbatim}
cfm(mu = 0.04, delta = log(1 + gl.g(i)), w = 1000)
\end{verbatim}
\end{Usage}
%
\begin{Arguments}
\begin{ldescription}
\item[\code{mu}] the force of mortality

\item[\code{delta}] the force of interest

\item[\code{w}] the arbitrarily large limiting age
\end{ldescription}
\end{Arguments}
%
\begin{Details}\relax
To revert to makehams use makehams()
\end{Details}
\inputencoding{utf8}
\HeaderA{createInsuranceTable}{Create Insurance Table}{createInsuranceTable}
%
\begin{Description}\relax
Creates a table containing EPV's of whole life insurances (discrete)
\end{Description}
%
\begin{Usage}
\begin{verbatim}
createInsuranceTable(x = gl.g(x), w = gl.g(w), d = gl.g(d), n = 5,
  i = gl.g(i), mt = 1)
\end{verbatim}
\end{Usage}
%
\begin{Arguments}
\begin{ldescription}
\item[\code{x}] the starting age

\item[\code{w}] the limiting age

\item[\code{d}] the select period

\item[\code{n}] pure endowment period

\item[\code{i}] the interest rate

\item[\code{mt}] the moment t calculate
\end{ldescription}
\end{Arguments}
%
\begin{Details}\relax
Computes life table using recursion
\end{Details}
\inputencoding{utf8}
\HeaderA{createLifeTable}{Create Ultimate Select Life Table}{createLifeTable}
%
\begin{Description}\relax
Creates a life table based on the select period, radix and Makeham model parameters
\end{Description}
%
\begin{Usage}
\begin{verbatim}
createLifeTable(x = gl.g(x), w = gl.g(w), radix = gl.g(radix),
  d = gl.g(d))
\end{verbatim}
\end{Usage}
%
\begin{Arguments}
\begin{ldescription}
\item[\code{x}] the starting age for the life table

\item[\code{w}] the limiting age

\item[\code{radix}] the number of individuals aged x

\item[\code{d}] the select period
\end{ldescription}
\end{Arguments}
%
\begin{Details}\relax
See Appendix Tables of DHW 2nd edition
\end{Details}
\inputencoding{utf8}
\HeaderA{demoivres}{Change Survival Model to DeMoivre's}{demoivres}
%
\begin{Description}\relax
Change Survival Model to DeMoivre's
\end{Description}
%
\begin{Usage}
\begin{verbatim}
demoivres(w = 100, delta = log(1 + gl.g(i)))
\end{verbatim}
\end{Usage}
%
\begin{Arguments}
\begin{ldescription}
\item[\code{w}] the limiting age

\item[\code{delta}] the force of interest
\end{ldescription}
\end{Arguments}
%
\begin{Details}\relax
Changes parameters and force of interest function to Uniform model
\end{Details}
\inputencoding{utf8}
\HeaderA{makehams}{Change Survival to Makeham's}{makehams}
%
\begin{Description}\relax
Change Survival to Makeham's
\end{Description}
%
\begin{Usage}
\begin{verbatim}
makehams(A = 0.00022, B = 2.7e-06, c = 1.124, d = 2, x = 20,
  w = 131, radix = 1e+05, i = 0.05)
\end{verbatim}
\end{Usage}
%
\begin{Arguments}
\begin{ldescription}
\item[\code{A}] model parameter

\item[\code{B}] model parameter

\item[\code{c}] model parameter

\item[\code{d}] select period

\item[\code{x}] the default age

\item[\code{w}] the limiting age

\item[\code{radix}] the number of starting individuals in life table

\item[\code{i}] the effective annual interest rate
\end{ldescription}
\end{Arguments}
%
\begin{Details}\relax
Reverts the survival model back to Makeham's law with default parameters
\end{Details}
\inputencoding{utf8}
\HeaderA{tEx}{Actuarial Present Value Factor}{tEx}
%
\begin{Description}\relax
Calculates the Expected Present value of a pure endowment insurance
\end{Description}
%
\begin{Usage}
\begin{verbatim}
tEx(t, x = gl.g(x), s = 0, i = gl.g(i), mt = 1)
\end{verbatim}
\end{Usage}
%
\begin{Arguments}
\begin{ldescription}
\item[\code{t}] the years from x

\item[\code{x}] the current age

\item[\code{s}] the select used so far

\item[\code{i}] the interest rate

\item[\code{mt}] the moment of the insurance
\end{ldescription}
\end{Arguments}
%
\begin{Details}\relax
Alternative actuarial "A" notation is also used for tEx
\end{Details}
\inputencoding{utf8}
\HeaderA{thV}{Benefit Reserve}{thV}
%
\begin{Description}\relax
Uses Euler's method to solve Thiele's differential equation to approximate the value of t+hV
\end{Description}
%
\begin{Usage}
\begin{verbatim}
thV(t = 0, h = 1, x = gl.g(x), tV = 0, Pt = function(t) t^0 *
  gl.g(pi), deltat = function(t) t^0 * log(1 + gl.g(i)), bt = function(t)
  t^0, ut = function(t) uxt(t, x), s = 0.01)
\end{verbatim}
\end{Usage}
%
\begin{Arguments}
\begin{ldescription}
\item[\code{t}] the time for which the reserve is known

\item[\code{h}] the the time from t for which the reserve should be calculated

\item[\code{x}] the age of the person for which the reserve is being calculated

\item[\code{tV}] the value of the reserve at time t

\item[\code{Pt}] the premium as a function of t

\item[\code{deltat}] the force of interest as a function of t

\item[\code{bt}] the death benefit payable immediately at the time of death as a function of t

\item[\code{ut}] the force of mortality as a function of t

\item[\code{s}] the step to use in Euler's method
\end{ldescription}
\end{Arguments}
%
\begin{Details}\relax
This function does not take into account expenses
\end{Details}
\inputencoding{utf8}
\HeaderA{tpx}{Survival Function}{tpx}
%
\begin{Description}\relax
Probability that x survives t years given survival to age x
\end{Description}
%
\begin{Usage}
\begin{verbatim}
tpx(t, x = gl.g(x), s = 0, uxt = gl.g(uxt))
\end{verbatim}
\end{Usage}
%
\begin{Arguments}
\begin{ldescription}
\item[\code{t}] the number of years to survive

\item[\code{x}] the current age

\item[\code{s}] select already used

\item[\code{uxt}] the force of mortality (can be used to override the default force of mortality)
\end{ldescription}
\end{Arguments}
%
\begin{Details}\relax
Uses a default select period of 2 (for makeham's law)
\end{Details}
\inputencoding{utf8}
\HeaderA{tqx}{CDF of Future Lifetime}{tqx}
%
\begin{Description}\relax
Probability that x dies in the next t years, given survival to age x
\end{Description}
%
\begin{Usage}
\begin{verbatim}
tqx(t, x = gl.g(x), s = 0, uxt = gl.g(uxt))
\end{verbatim}
\end{Usage}
%
\begin{Arguments}
\begin{ldescription}
\item[\code{t}] the number of years before death

\item[\code{x}] the current age

\item[\code{s}] select already used

\item[\code{uxt}] the force of mortality (can be used to override the default force of mortality)
\end{ldescription}
\end{Arguments}
%
\begin{Details}\relax
Calcualted as 1 - tpx(t,x)
\end{Details}
\inputencoding{utf8}
\HeaderA{udeferredtqx}{Deferred CDF of Future Lifetime}{udeferredtqx}
%
\begin{Description}\relax
Probability of surviving u years and dying in the next t years
\end{Description}
%
\begin{Usage}
\begin{verbatim}
udeferredtqx(u, t = 1, x = gl.g(x), s = 0)
\end{verbatim}
\end{Usage}
%
\begin{Arguments}
\begin{ldescription}
\item[\code{u}] the number of years to survive

\item[\code{t}] the number of years to death within

\item[\code{x}] the current age

\item[\code{s}] the select used
\end{ldescription}
\end{Arguments}
%
\begin{Details}\relax
Can be calculated by splitting the CDF. Use tpx(u,x) - tpx(u+t,x)
\end{Details}
\inputencoding{utf8}
\HeaderA{uxt}{Force of Mortality}{uxt}
%
\begin{Description}\relax
The select force of mortality, u[x]+s = 0.9\textasciicircum{}(2-s) ux+s
where the force of mortality is ux+s = A + Bc\textasciicircum{}(x+t)
\end{Description}
%
\begin{Usage}
\begin{verbatim}
uxt(t, x = gl.g(x), s = 0, d = gl.g(d), A = gl.g(A), B = gl.g(B),
  c = gl.g(c))
\end{verbatim}
\end{Usage}
%
\begin{Arguments}
\begin{ldescription}
\item[\code{t}] the years after age x

\item[\code{x}] the current age

\item[\code{s}] select already used

\item[\code{d}] the select period

\item[\code{A}] Makeham model parameter

\item[\code{B}] Makeham model parameter

\item[\code{c}] Makeham model parameter
\end{ldescription}
\end{Arguments}
\inputencoding{utf8}
\HeaderA{v}{Present Value Factor}{v}
%
\begin{Description}\relax
Calculates the present value of a cash flow
\end{Description}
%
\begin{Usage}
\begin{verbatim}
v(i = gl.g(i), n = 1, delta = log(1 + i))
\end{verbatim}
\end{Usage}
%
\begin{Arguments}
\begin{ldescription}
\item[\code{i}] the effective annual interest rate

\item[\code{n}] the number of years to apply discounting

\item[\code{delta}] the force of interest
\end{ldescription}
\end{Arguments}
%
\begin{Details}\relax
The force of interest is internally derived from the effective annual interest rate
\end{Details}
\printindex{}
\end{document}
